%% Nix Quick Reference Sheet
%% James Geddes, The Alan Turing Institute, 2023
%% CC0
\documentclass[9pt, a4paper, landscape]{extarticle}
\usepackage[T1]{fontenc}
\usepackage{multicol}
\setlength{\columnsep}{3em}
\usepackage[medium, compact]{titlesec}
\titleformat{\section}[block]{\Large\bfseries\filcenter}{\thesection}{1em}{}
\usepackage{beton}
\DeclareFontSeriesDefault[rm]{bf}{sbc}
\usepackage{eulervm}
\usepackage{amsmath}
\usepackage[margin=0.51in]{geometry}
\usepackage{parskip}
\usepackage{tabularx}
\usepackage{array}
\usepackage{booktabs}
\usepackage{microtype}
%\usepackage{fancyhdr}
%%
%%\usepackage[style=authoryear]{biblatex}
%%\addbibresource{notes.bib}
%%
%%
\newcommand{\cmd}[1]{\texttt{#1}}
\newcommand{\str}{\textsf{String}}
\newcommand{\pth}{\textsf{Path}}
\newcommand{\itg}{\textsf{Int}} % \int was taken
\newcommand{\flt}{\testsf{Float}}
\newcommand{\num}{\textsf{Num}} % Either an int or a float
\newcommand{\lst}{\textsf{List}}
\newcommand{\ats}{\textsf{Attrs}}
\newcommand{\bln}{\textsf{Bool}}

%%
\title{Nix Quick Reference}
%%\author{}
\date{\vskip-10ex November 2023}
%%
\renewcommand{\arraystretch}{1.2}
\setlength{\tabcolsep}{0.5em}
%%
\begin{document}%\maketitle
%%
\begin{multicols*}{3}\raggedcolumns%
%%
\maketitle

\section*{Builtins}

%% ============================================================
%% Operators

\subsection*{Operators}

The following are approximately in order of precendence. Function
application (represented by a space) has precedence just below
attribute selection.

The expression \emph{attrs} denotes an atrribute set (that is, a map
from strings to values). An \emph{attrpath} is a dot-separated list of
attribute names.

\begin{tabularx}{\columnwidth}{@{}l>{\raggedright\arraybackslash}X@{}}

  \parbox[t]{8em}{\emph{attrs}\texttt{.}\emph{attrpath} \\
    ${}\;\;[$ \texttt{or} \emph{expr} $]$}
  & Attribute selection (with default if attribute does not exist). \\

  \emph{attrs} \texttt{?} \emph{attrpath} & Test whether attribute
  exists. \\

  \cmd{++} & List concatenation. \\
  
  \cmd{*}, \cmd{/}, \cmd{-}, \cmd{+} & Arithmetic (including unary
  minus, which has higher precedence than attribute testing). \\
  
  \cmd{+} & String (and path) concatenation. \\

  \emph{attrs} \cmd{//} \emph{attrs} & Update attribute set on the
  left with the entries from the attribute set on the right. \\
  
  \cmd{<}, \cmd{<=}, \cmd{>}, \cmd{>=}, \cmd{==}, \cmd{!=} &
  Comparison. \\

  \cmd{!}, \cmd{\&\&}, \cmd{||}, \cmd{->} & Logical operators,
  including negation and implication. (Negation has higher precendence
  than comparison.) \\
\end{tabularx}

%% ============================================================
%% Constants

\columnbreak
\subsection*{Constants}
Specific values show the value of the constant as of today, on my
machine.

\begin{tabularx}{\columnwidth}{@{}l>{\raggedright\arraybackslash}X@{}}
  \cmd{currentSystem} & `\cmd{aarch64-darwin}'. \\
  \cmd{currentTime} & `1700936742'. Unix time; does not update on repeated
  evaluation. \\
  \cmd{nixVersion} & `\cmd{2.18.1}', as returned by `cmd{nix -{}-version}'. \\
  \cmd{langVersion} & `6'. \\
  \cmd{builtins} & \cmd{\{\ ...\ \}}. This very list! \\
  \cmd{storeDir} & `\cmd{/nix/store}'. \\
  \cmd{nixPath} & `\cmd{[\ ...\ ]}'. List of paths to search when a path is
  written in the form `\cmd{<\emph{path}>}'.  \\
  \cmd{derivation} & \cmd{«lambda @ /builtin/derivation.nix:5:1»} No
  idea! \\
  \cmd{true}, \cmd{false} & Boolean values. \\
  \cmd{null} & The value `\cmd{null}'. \\
\end{tabularx}

%% ============================================================
%% Arithmetic and logical

\subsection*{Arithmetic and logical}

\begin{tabularx}{\columnwidth}{@{}l>{\raggedright\arraybackslash}X@{}}
  \cmd{isInt}, \cmd{isFloat} & $\alpha\to\bln$. Predicates for numbers. \\

  \cmd{isBool} & $\alpha\to\bln$. Predicate for boolean. \\

  \cmd{add}, \cmd{sub}, \cmd{mul}, \cmd{div} & Binary operators. \\

  \cmd{ceil}, \cmd{floor} & $\num\to\itg$. \\

  \cmd{lessThan} & Comparison. \\

  \cmd{bitAnd}, \cmd{bitOr}, \cmd{bitXor} & Bitwise logical
  operators. \\
\end{tabularx}

%% ============================================================
%% Other types

\subsection*{Other types}

\begin{tabularx}{\columnwidth}{@{}l>{\raggedright\arraybackslash}X@{}}
  \cmd{isFunction} & $\alpha\to\bln$. Predicate for functions. \\

  \cmd{isNull} & $\alpha\to\bln$. Predicate for value `\cmd{null}'. \\

  \cmd{typeOf} & $\alpha\to\str$. Return string representation of type of
  value. (Attribute sets return `\cmd{set}'; functions return
  `\cmd{lambda}'.) \\
  
\end{tabularx}


%% ============================================================
%% Lists

\subsection*{Lists}
\begin{tabularx}{\columnwidth}{@{}l>{\raggedright\arraybackslash}X@{}}
  \cmd{isList}      & $\alpha\to\bln$. Predicate for lists. \\
  \cmd{[]}          & The empty list. \\
  \cmd{genList}     & $(\itg\to\alpha)\to\itg\to\lst[\alpha]$.\newline
  \cmd{genList} $f$ $n$ applies $f$ to the list 0, 1, 2, \dots, $n-1$. \\

  \cmd{head}, \cmd{tail} & First and rest. \textbf{Warning:} \cmd{tail}
  is $O(N)$! \\
  \cmd{elemAt}      & $\lst[\alpha] \to \itg \to \alpha$. Get the $n$th element of a list. \\
  \cmd{length}      & $\lst \to \itg$. \\
  
  \cmd{map}         & $(\alpha\to\beta)\to\lst[\alpha]\to\lst[\beta]$. \\
  \cmd{filter}      & $(\alpha\to\bln)\to\lst[\alpha]\to\lst[\alpha]$. \\
  \cmd{foldl'}      & $(\beta\to\alpha\to\beta)\to\beta\to\lst[\alpha]\to\beta$. Reduce the list.\\
  
  \cmd{elem}        & $\alpha\to\lst[\alpha]\to\bln$. Is an element in the list? \\
  \cmd{all}         & $(\alpha \to \bln) \to \lst[\alpha] \to \bln$. \\
  \cmd{any}         & $(\alpha \to \bln) \to \lst[\beta] \to \bln$. \\ 
  
  \cmd{concatLists} & $\lst[\lst[\alpha]]\to\lst[\alpha]$.\\
  \cmd{concatMap}   & $(\alpha\to\lst[\beta])\to\lst[\alpha]\to\lst[\beta]$. \cmd{concatMap f
    xs} is equivalent to \cmd{concatLists (map f xs)} but faster. \\

  \cmd{groupBy}     & $(\alpha\to\str)\to\lst[\alpha]\to\ats$. Returns an attribute
  set whose names are computed from each list element and whose values
  are sublists. \\
  \cmd{partition}   & $(\alpha\to\bln)\to\lst[\alpha]\to\ats$. Returns an attribute set
  with two names, \cmd{right} and \cmd{wrong}, and sublists as values. \\

  \cmd{sort}        & $(\alpha\to\alpha\to\bln)\to\lst[\alpha]\to\lst[\alpha]$. \\

  \cmd{listToAttrs} & $\lst[\ats]\to\ats$. Construct an attribute set
  from a list of attribute sets, each having exactly two names,
  \cmd{name} and \cmd{value}. \\
\end{tabularx}

%% ============================================================
%% Attribute sets

\subsection*{Attribute sets}
\begin{tabularx}{\columnwidth}{@{}l>{\raggedright\arraybackslash}X@{}}
  \cmd{isAttrs} & $\alpha\to\bln$. Predicate for attribute sets. \\

  \cmd{attrNames} & $\ats \to \lst$. Names, alphabetically sorted. \\

  \cmd{attrValues} & $\ats \to \lst$. Values, in the same order as
  returned by\cmd{attrNames}. \\

  \cmd{catAttrs} & $\str \to \lst[\ats] \to \lst$. Return the values of a
  named attribute from a list of attribute sets. \\

  \cmd{functionArgs} & $(\alpha\to\beta)\to\ats$. Return names of attributes
  pattern-matched by function. Values are a boolean indicating whether
  argument has a default value. \\

  \cmd{genericClosure} & $\ats\{\cmd{startSet},
  \cmd{operator}\}\to\lst[\ats]$. Repeatedly apply \cmd{operator} to
  each attribute set in a $\lst[\ats]$ that contains a value

  \cmd{key}, strating with \cmd{startSet} and producing a new
  attribute set with a \cmd{key}, until no more are produced. \\

  \cmd{getAttr} & $\str\to\ats\to\alpha$. Return value of named attribute. \\

  \cmd{hasAttr} & $\str\to\ats\to\bln$. Check whether named attribute
  exists. \\

  \cmd{intersectAttrs} & $\ats\to\ats\to\ats$. Intersect by name; values
  in second argument take precedence. \\

  \cmd{mapAttrs} & $(\str\to\alpha\to\beta)\to\ats\to\ats$. Map over attribute set. \\

  \cmd{zipAttrsWith} & $(\str\to\lst[\alpha]\to\beta)\to\lst[\ats]\to\ats$. Transpose
  list of attribute sets into an attribute set of lists and apply
  \cmd{mapAttrs}.\\

  \cmd{removeAttrs} & $\ats\to\lst[\str]\to\ats$. Remove attributes. \\

\end{tabularx}

%% ============================================================
%% Strings

\subsection*{Strings}
\begin{tabularx}{\columnwidth}{@{}l>{\raggedright\arraybackslash}X@{}}

  \cmd{isString} & $\alpha\to\bln$. Predicate for strings. \\

  
  \cmd{concatStringsSep} & $\str\to\lst[\str]\to\str$. Concactenate a list
  of strings, splicing in a separator. \\

  \cmd{convertHash} & $\ats\to\str$. [2.19] Convert a hash to specified
  format. The attribute set must contain names `\cmd{hash}',
  `\cmd{hashAlgo}', and `\cmd{toHashFormat}'. \\

  \cmd{fromJSON}, \cmd{fromTOML} & $\str\to\alpha$. Convert string to a Nix
  value. \\

  \cmd{hashString} & $\str\to\str\to\str$. Hash the second argument using
  the method indicated by the first argument. \\
  
  \cmd{match} & $\str\to\str\to\lst[\str]$. Regular expression
  matching. First argument is extended POSIX regex. \\
  
  \cmd{split} & $\str\to\str\to\lst[\str]$. List of matches in second
  argument by regex groups given in first, interleaving with
  non-matched strings. \\
    
  \cmd{replaceStrings} & $\lst[\str]\to\lst[\str]\to\str$. Replace every
  occurence of substring in first argument with string in second. \\

  \cmd{stringLength} & $\str\to\itg$. \\

  \cmd{substring} & $\itg\to\itg\to\str$. Arguments are start and
  length. A length of $-1$ means ``until end of string.'' \\

  \cmd{toJSON} & $\alpha\to\str$. Convert to JSON.\@ Derivations are translated
  to output paths; paths are copied to the store and converted to
  store paths. \\
  
  \cmd{toXML} & $\alpha\to\str$. Convert to XML representation. \\

  \cmd{toString} & $\alpha\to\str$. Booleans become \cmd{""} and
  \cmd{"1"}. Attribute sets use the value of applying
  \cmd{\_\_toString} to themselves or the value of \cmd{outPath}. \\

\end{tabularx}

%% ============================================================
%% Paths and files

\subsection{Paths and files}
\begin{tabularx}{\columnwidth}{@{}l>{\raggedright\arraybackslash}X@{}}
  \cmd{isPath} & $\alpha\to\bln$. Predicate for paths. \\

  \cmd{baseNameOf} & $\str\to\str$. Everything following the last
  `\cmd{/}' in a path. \\

  \cmd{dirOf} & $\str\to\str$. Everything before the final `\cmd{/}`. \\
  
  \cmd{compareVersions} & $\str\to\str\to\bln$. Compare versions
  represented as semantic version strings. \\
  
  \cmd{splitVersion} & $\str\to\lst[\str]$. Split semantic version into
  components. \\

  
  \cmd{toPath} & Deprecated. Use path concatenation. \\
\end{tabularx}

%% ============================================================
%% Other

\subsection*{Control flow}

\begin{tabularx}{\columnwidth}{@{}l>{\raggedright\arraybackslash}X@{}}
  \cmd{abort} & $\str \to \bot $. Abort and print message. \\

  \cmd{break} & $\str \to \bot $. In debug mode, enter REPL.\@ Otherwise
  print string.\\

  \cmd{seq}  & $\cmd{seq} \; e_1 \; e_2$ evaluates $e_1$ and returns $e_2$.  \\

  \cmd{deepSeq} & Like \cmd{seq} except that elements or attributes of
  the first expression are recursively evaluted. \\ 
  
  \cmd{throw} & $\str\to\bot$. Throw error message. Occasionally does not
  abort. \\

  \cmd{trace} & $\alpha\to\beta\to\beta$. Print first argument on stderr, then evaluate
  second argument. \\

  \cmd{traceVerbose} &  $\alpha\to\beta\to\beta$. Like \cmd{trace} but only print if
  \cmd{{-}{-}trace-verbose} enabled. \\

  \cmd{tryEval} & $\alpha\to\ats\{\cmd{success}; \cmd{value}\}$. Attempt to
  shallowly evaluate expression, catching \cmd{throw} or \cmd{assert}
  (but not \cmd{abort}). 
\end{tabularx}


\cmd{placeholder} 

\cmd{derivationStrict}:

\cmd{fetchGit}:
\cmd{fetchMercurial}:
\cmd{fetchTarball}:
\cmd{fetchTree}:
\cmd{fetchurl}:

\cmd{filterSource}:
\cmd{findFile}:

\cmd{flakeRefToString}:

\cmd{getContext}:
\cmd{getEnv}:
\cmd{getFlake}:
\cmd{hasContext}:

\cmd{hashFile}:
\cmd{import}:

\cmd{parseDrvName}:
\cmd{parseFlakeRef}:


\cmd{path}:
\cmd{pathExists}:

\cmd{readDir}:
\cmd{readFile}:
\cmd{readFileType}:

\cmd{scopedImport}:



\cmd{storePath}:

\cmd{toFile}:



\cmd{unsafeDiscardOutputDependency}:
\cmd{unsafeDiscardStringContext}:

\cmd{unsafeGetAttrPos}:


\subsection*{Derivations}

% \begin{tabularx}{\columnwidth}{@{}l>{\raggedright\arraybackslash}X@{}}
  \cmd{addDrvOutputDependencies} ?

    Create a copy of the given string where a single consant string context element is turned into a "derivation deep" string context element.

    The store path that is the constant string context element should point to a valid derivation, and end in .drv.

    The original string context element must not be empty or have multiple elements, and it must not have any other type of element other than a constant or derivation deep element. The latter is supported so this function is idempotent.

    This is the opposite of builtins.unsafeDiscardOutputDependency.
  
\cmd{addErrorContext} ?
\cmd{appendContext} ?

% \end{tabularx}

\end{multicols*}
\end{document}
