%% Nix Quick Reference Sheet
%% James Geddes, The Alan Turing Institute, 2023
%% CC0
\documentclass[9pt, a4paper, landscape]{extarticle}
\usepackage[T1]{fontenc}
\usepackage{multicol}
\setlength{\columnsep}{3em}
\usepackage[medium, compact]{titlesec}
\titleformat{\section}[block]{\Large\bfseries\filcenter}{\thesection}{1em}{}
\usepackage{beton}
\DeclareFontSeriesDefault[rm]{bf}{sbc}
\usepackage{eulervm}
\usepackage{amsmath}
\usepackage[margin=0.51in]{geometry}
\usepackage{parskip}
\usepackage{tabularx}
\usepackage{array}
\usepackage{booktabs}
\usepackage{microtype}
%\usepackage{fancyhdr}
%%
%%\usepackage[style=authoryear]{biblatex}
%%\addbibresource{notes.bib}
%%
%%
\newcommand{\cmd}[1]{\texttt{#1}}
\newcommand{\str}{\textsf{String}}
\newcommand{\pth}{\textsf{Path}}
\newcommand{\itg}{\textsf{Int}} % \int was taken
\newcommand{\flt}{\testsf{Float}}
\newcommand{\num}{\textsf{Num}} % Either an int or a float
\newcommand{\lst}{\textsf{List}}
\newcommand{\ats}{\textsf{Attrs}}
\newcommand{\bln}{\textsf{Bool}}

%%
\title{Nix Builtins Quick Reference}
%%\author{}
\date{\vskip-10ex August 2024}
%%
\renewcommand{\arraystretch}{1.2}
\setlength{\tabcolsep}{0.5em}
%%
\begin{document}%\maketitle
%%
\begin{multicols*}{3}\raggedcolumns%
%%
\maketitle

\section*{Builtins}

%% ============================================================
%% Operators

\subsection*{Operators}

The following are approximately in order of precendence. Function
application (represented by a space) has precedence just below
attribute selection.

The expression \emph{attrs} denotes an atrribute set (that is, a map
from strings to values). An \emph{attrpath} is a dot-separated list of
attribute names.

\begin{tabularx}{\columnwidth}{@{}l>{\raggedright\arraybackslash}X@{}}

  \parbox[t]{8em}{\emph{attrs}\texttt{.}\emph{attrpath} \\
    ${}\;\;[$ \texttt{or} \emph{expr} $]$}
  & Attribute selection (with default if attribute does not exist). \\

  \emph{attrs} \texttt{?} \emph{attrpath} & Test whether attribute
  exists. \\

  \cmd{++} & List concatenation. \\
  
  \cmd{*}, \cmd{/}, \cmd{-}, \cmd{+} & Arithmetic (including unary
  minus, which has higher precedence than attribute testing). \\
  
  \cmd{+} & String (and path) concatenation. \\

  \emph{attrs} \cmd{//} \emph{attrs} & Update attribute set on the
  left with the entries from the attribute set on the right. \\
  
  \cmd{<}, \cmd{<=}, \cmd{>}, \cmd{>=}, \cmd{==}, \cmd{!=} &
  Comparison. \\

  \cmd{!}, \cmd{\&\&}, \cmd{||}, \cmd{->} & Logical operators,
  including negation and implication. (Negation has higher precendence
  than comparison.) \\
\end{tabularx}

%% ============================================================
%% Constants

\columnbreak
\subsection*{Constants}
Specific values show the value of the constant as of today, on my
machine.

\begin{tabularx}{\columnwidth}{@{}l>{\raggedright\arraybackslash}X@{}}
  \cmd{currentSystem} & `\cmd{aarch64-darwin}'. \\
  \cmd{currentTime} & `1700936742'. Unix time; does not update on repeated
  evaluation. \\
  \cmd{nixVersion} & `\cmd{2.18.1}', as returned by `cmd{nix -{}-version}'. \\
  \cmd{langVersion} & `6'. \\
  \cmd{builtins} & \cmd{\{\ ...\ \}}. This very list! \\
  \cmd{storeDir} & `\cmd{/nix/store}'. \\
  \cmd{nixPath} & `\cmd{[\ ...\ ]}'. List of paths to search when a path is
  written in the form `\cmd{<\emph{path}>}'.  \\
  \cmd{true}, \cmd{false} & Boolean values. \\
  \cmd{null} & The value `\cmd{null}'. \\
\end{tabularx}

%% ============================================================
%% Arithmetic and logical

\subsection*{Arithmetic and logical}

\begin{tabularx}{\columnwidth}{@{}l>{\raggedright\arraybackslash}X@{}}
  \cmd{isInt}, \cmd{isFloat} & $\alpha\to\bln$. Predicates for numbers. \\

  \cmd{isBool} & $\alpha\to\bln$. Predicate for boolean. \\

  \cmd{add}, \cmd{sub}, \cmd{mul}, \cmd{div} & Binary operators. \\

  \cmd{ceil}, \cmd{floor} & $\num\to\itg$. \\

  \cmd{lessThan} & Comparison. \\

  \cmd{bitAnd}, \cmd{bitOr}, \cmd{bitXor} & Bitwise logical
  operators. \\
\end{tabularx}

%% ============================================================
%% Other types

\subsection*{Other types}

\begin{tabularx}{\columnwidth}{@{}l>{\raggedright\arraybackslash}X@{}}
  \cmd{isFunction} & $\alpha\to\bln$. Predicate for functions. \\

  \cmd{isNull} & $\alpha\to\bln$. Predicate for value `\cmd{null}'. \\

  \cmd{typeOf} & $\alpha\to\str$. Return string representation of type of
  value. (Attribute sets return `\cmd{set}'; functions return
  `\cmd{lambda}'.) \\

  \cmd{getEnv} & $\str\to\alpha$. Returns the value of an environment variable. 

  
\end{tabularx}


%% ============================================================
%% Lists

\subsection*{Lists}
\begin{tabularx}{\columnwidth}{@{}l>{\raggedright\arraybackslash}X@{}}
  \cmd{isList}      & $\alpha\to\bln$. Predicate for lists. \\
  \cmd{[]}          & The empty list. \\
  \cmd{genList}     & $(\itg\to\alpha)\to\itg\to\lst[\alpha]$.\newline
  \cmd{genList} $f$ $n$ applies $f$ to the list 0, 1, 2, \dots, $n-1$. \\

  \cmd{head}, \cmd{tail} & First and rest. \textbf{Warning:} \cmd{tail}
  is $O(N)$! \\
  \cmd{elemAt}      & $\lst[\alpha] \to \itg \to \alpha$. Get the $n$th element of a list. \\
  \cmd{length}      & $\lst \to \itg$. \\
  
  \cmd{map}         & $(\alpha\to\beta)\to\lst[\alpha]\to\lst[\beta]$. \\
  \cmd{filter}      & $(\alpha\to\bln)\to\lst[\alpha]\to\lst[\alpha]$. \\
  \cmd{foldl'}      & $(\beta\to\alpha\to\beta)\to\beta\to\lst[\alpha]\to\beta$. Reduce the list.\\
  
  \cmd{elem}        & $\alpha\to\lst[\alpha]\to\bln$. Is an element in the list? \\
  \cmd{all}         & $(\alpha \to \bln) \to \lst[\alpha] \to \bln$. \\
  \cmd{any}         & $(\alpha \to \bln) \to \lst[\beta] \to \bln$. \\ 
  
  \cmd{concatLists} & $\lst[\lst[\alpha]]\to\lst[\alpha]$.\\
  \cmd{concatMap}   & $(\alpha\to\lst[\beta])\to\lst[\alpha]\to\lst[\beta]$. \cmd{concatMap f
    xs} is equivalent to \cmd{concatLists (map f xs)} but faster. \\

  \cmd{groupBy}     & $(\alpha\to\str)\to\lst[\alpha]\to\ats$. Returns an attribute
  set whose names are computed from each list element and whose values
  are sublists. \\
  \cmd{partition}   & $(\alpha\to\bln)\to\lst[\alpha]\to\ats$. Returns an attribute set
  with two names, \cmd{right} and \cmd{wrong}, and sublists as values. \\

  \cmd{sort}        & $(\alpha\to\alpha\to\bln)\to\lst[\alpha]\to\lst[\alpha]$. \\

  \cmd{listToAttrs} & $\lst[\ats]\to\ats$. Construct an attribute set
  from a list of attribute sets, each having exactly two names,
  \cmd{name} and \cmd{value}. \\
\end{tabularx}

%% ============================================================
%% Attribute sets

\subsection*{Attribute sets}
\begin{tabularx}{\columnwidth}{@{}l>{\raggedright\arraybackslash}X@{}}
  \cmd{isAttrs} & $\alpha\to\bln$. Predicate for attribute sets. \\

  \cmd{attrNames} & $\ats \to \lst$. Names, alphabetically sorted. \\

  \cmd{attrValues} & $\ats \to \lst$. Values, in the same order as
  returned by\cmd{attrNames}. \\

  \cmd{catAttrs} & $\str \to \lst[\ats] \to \lst$. Return the values of a
  named attribute from a list of attribute sets. \\

  \cmd{functionArgs} & $(\alpha\to\beta)\to\ats$. Return names of attributes
  pattern-matched by function. Values are a boolean indicating whether
  argument has a default value. \\

  \cmd{genericClosure} & $\ats\{\cmd{startSet},
  \cmd{operator}\}\to\lst[\ats]$. Repeatedly apply \cmd{operator} to
  each attribute set in a $\lst[\ats]$ that contains a value

  \cmd{key}, strating with \cmd{startSet} and producing a new
  attribute set with a \cmd{key}, until no more are produced. \\

  \cmd{getAttr} & $\str\to\ats\to\alpha$. Return value of named attribute. \\

  \cmd{hasAttr} & $\str\to\ats\to\bln$. Check whether named attribute
  exists. \\

  \cmd{intersectAttrs} & $\ats\to\ats\to\ats$. Intersect by name; values
  in second argument take precedence. \\

  \cmd{mapAttrs} & $(\str\to\alpha\to\beta)\to\ats\to\ats$. Map over attribute set. \\

  \cmd{zipAttrsWith} & $(\str\to\lst[\alpha]\to\beta)\to\lst[\ats]\to\ats$. Transpose
  list of attribute sets into an attribute set of lists and apply
  \cmd{mapAttrs}.\\

  \cmd{removeAttrs} & $\ats\to\lst[\str]\to\ats$. Remove attributes. \\

\end{tabularx}

%% ============================================================
%% Strings

\subsection*{Strings}
\begin{tabularx}{\columnwidth}{@{}l>{\raggedright\arraybackslash}X@{}}

  \cmd{isString} & $\alpha\to\bln$. Predicate for strings. \\

  
  \cmd{concatStringsSep} & $\str\to\lst[\str]\to\str$. Concactenate a list
  of strings, splicing in a separator. \\

  \cmd{convertHash} & $\ats\to\str$. [2.19] Convert a hash to specified
  format. The attribute set must contain names `\cmd{hash}',
  `\cmd{hashAlgo}', and `\cmd{toHashFormat}'. \\

  \cmd{fromJSON}, \cmd{fromTOML} & $\str\to\alpha$. Convert string to a Nix
  value. \\

  \cmd{hashString} & $\str\to\str\to\str$. Hash the second argument using
  the method indicated by the first argument. \\
  
  \cmd{match} & $\str\to\str\to\lst[\str]$. Regular expression
  matching. First argument is extended POSIX regex. \\
  
  \cmd{split} & $\str\to\str\to\lst[\str]$. List of matches in second
  argument by regex groups given in first, interleaving with
  non-matched strings. \\
    
  \cmd{replaceStrings} & $\lst[\str]\to\lst[\str]\to\str$. Replace every
  occurence of substring in first argument with string in second. \\

  \cmd{stringLength} & $\str\to\itg$. \\

  \cmd{substring} & $\itg\to\itg\to\str$. Arguments are start and
  length. A length of $-1$ means ``until end of string.'' \\

  \cmd{toJSON} & $\alpha\to\str$. Convert to JSON.\@ Derivations are translated
  to output paths; paths are copied to the store and converted to
  store paths. \\
  
  \cmd{toXML} & $\alpha\to\str$. Convert to XML representation. \\

  \cmd{toString} & $\alpha\to\str$. Booleans become \cmd{""} and
  \cmd{"1"}. Attribute sets use the value of applying
  \cmd{\_\_toString} to themselves or the value of \cmd{outPath}. \\

\end{tabularx}

%% ============================================================
%% Paths and files

\subsection{Paths and files}
\begin{tabularx}{\columnwidth}{@{}l>{\raggedright\arraybackslash}X@{}}
  \cmd{isPath} & $\alpha\to\bln$. Predicate for paths. \\

  \cmd{baseNameOf} & $\str\to\str$. Everything following the last
  `\cmd{/}' in a path. \\

  \cmd{dirOf} & $\str\to\str$. Everything before the final `\cmd{/}`. \\
  
  \cmd{findFile} & Find files, with complicated rules. Example:
                   \cmd{<nixpkgs>} is equivalent to
                   \cmd{builtins.findFile builtins.nixPath
                   "nixpkgs"}. \\
  
  \cmd{hashFile} & \cmd{hashFile type p} is a base~16 representation
                   of the hash of \cmd{p} using algorithm \cmd{type}. \\
  
  \cmd{compareVersions} & $\str\to\str\to\bln$. Compare versions
  represented as semantic version strings. \\
  
  \cmd{splitVersion} & $\str\to\lst[\str]$. Split semantic version into
  components. \\

  \cmd{path} & An enrichment of the built-in path type. TODO:\ I don't
               know what this means.  \\
  \cmd{pathExists} & $\str\to\bln$. True if the path exists when this
                     expressio is evaluated. \\
  \cmd{readDir} & $\str\to\ats$. Return a map from directory entries to
                  filetype. \\
  \cmd{readFile} & $\str\to\str$. Return the contents of the file as a
                   string. \\
  \cmd{readFileType} & $\str\to\str$. Return the filetype of the
                       file. One of \cmd{regular}, \cmd{director},
                       \cmd{symlink}, and \cmd{unknown}. \\

  
  \cmd{toPath} & Deprecated. Use path concatenation. \\
\end{tabularx}

%% ============================================================
%% Other

\subsection*{Control flow}

\begin{tabularx}{\columnwidth}{@{}l>{\raggedright\arraybackslash}X@{}}
  \cmd{abort} & $\str \to \bot $. Abort and print message. Can be
                used without the \cmd{builtins} prefix. \\

  \cmd{break} & $\str \to \bot $. In debug mode, enter REPL.\@ Otherwise
  print string.\\

  \cmd{seq}  & $\cmd{seq} \; e_1 \; e_2$ evaluates $e_1$ and returns $e_2$.  \\

  \cmd{deepSeq} & Like \cmd{seq} except that elements or attributes of
  the first expression are recursively evaluted. \\ 
  
  \cmd{import} & Parse and return the Nix expression in the file given
                 by its argument. (If a directory, then use
                 \cmd{default.nix} in that directory.) The expression
                 in the file cannot contain free variables but it can
                 evaluate to a function which may be applied. Can be
                 used without the \cmd{builtins} prefix. \\

  \cmd{throw} & $\str\to\bot$. Throw error message. Occasionally does not
                abort. Can be used without the \cmd{builtins} prefix. \\

  \cmd{trace} & $\alpha\to\beta\to\beta$. Print first argument on stderr, then evaluate
  second argument. \\

  \cmd{traceVerbose} &  $\alpha\to\beta\to\beta$. Like \cmd{trace} but only print if
  \cmd{{-}{-}trace-verbose} enabled. \\

  \cmd{tryEval} & $\alpha\to\ats\{\cmd{success}; \cmd{value}\}$. Attempt to
  shallowly evaluate expression, catching \cmd{throw} or \cmd{assert}
  (but not \cmd{abort}). 
\end{tabularx}

%% ============================================================
%% Derivations and the Store

\subsection*{Derivations and the Store}

\begin{tabularx}{\columnwidth}{@{}l>{\raggedright\arraybackslash}X@{}}
  \cmd{derivation} & Can be used with the \cmd{builtins} prefix. \\
  \cmd{placeholder} & $\str\to\str$. Typical arguments are \cmd{"out"},
                      \cmd{"bin"}, or \cmd{"dev"}. Return a ``placeholder string'' for
                      the output specified that will be substituted by
                      the corresponding output path at build
                      time. TODO:\ I don't know what this means. \\
  
  \cmd{fetchClosure} & $\ats\to\str$. Fetch a store path closure from a
                       binary cache and return the store path as a string with
                       context. TODO:\ I don't know what this means. \\
  \cmd{storePath} & Set a derivation attribute to the result of this
                    to induce a dependency on an existing store path
                    without copying its contents. \\
  \cmd{toFile} & \cmd{toFile name s} stores string \cmd{s} in a file
                 in the Nix store, the file having suffix \cmd{name},
                 and returns its path.  \\
                 
  \cmd{fetchGit} & Fetch a path from git. TODO:\ I don't know what
                   kind of value is returned. Probably the path to
                   the downloaded files? \\

  \cmd{fetchMercurial} & As above. \\
  \cmd{fetchTarball} & As above. \\
  \cmd{fetchTree} & As above. \\ 
  \cmd{fetchurl}: & As above. \\

  \cmd{filterSource} & Copy files to the Nix store while filtering
                       those files. \\
  
  \cmd{getContext} & $\str\to\ats$. Return the ``string context'' of its
                   argument: a map from store derivation paths to
                   output names. \\
  \cmd{hasContext} & $\str\to\bln$. True if the argument string has a
                   non-empty context. \\  

  \cmd{outputOf} & Return the output path of a derivation. TODO:\ I
                   don't know what this means. \\

  \cmd{parseDrvName} & $\str\to\ats\{name, version\}$. Split argument
                       into a package name and a version. \\
  
\end{tabularx}

%% ============================================================
%% Flakes

\subsection*{Flakes}

\begin{tabularx}{\columnwidth}{@{}l>{\raggedright\arraybackslash}X@{}}
\cmd{flakeRefToString} & Convert a flake reference from attribute set
                         format to URL format. \\
  
\cmd{getFlake} & Fetch a flake from a flake reference and return its
                 output attributes and some metadata. \\
  
\cmd{parseFlakeRef} & $\str\to\ats$. Parse a flake reference and return
                      its ``exploded form.'' \\

\end{tabularx}


\subsection*{Not in the manual}

\cmd{addErrorContext}, \cmd{appendContext},
\cmd{addDrvOutputDependencies}, \cmd{scopedImport}, \cmd{derivationStrict},
\cmd{unsafeDiscardOutputDependency}, \cmd{unsafeDiscardStringContext},
\cmd{unsafeGetAttrPos}.


\end{multicols*}
\end{document}

